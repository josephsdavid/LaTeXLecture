%\documentclass[10pt,a4paper]{article}
\documentclass[runningheads]{llncs}
\usepackage{blindtext}
\usepackage{lipsum}
\usepackage{xstring}
% https://tex.stackexchange.com/a/26808/2595
\makeatletter
\def\unpacklipsum#1#2#3{%
	\count@=#1\relax
	\advance\count@\m@ne
	\def#3{}%
	\loop\ifnum\count@<#2\relax
	\advance\count@\@ne
	\edef#3{#3\csname lipsum@\romannumeral\count@\endcsname}%
	\repeat}
\makeatother

\def\loremnchars[#1]#2{%
	\unpacklipsum{#1}{#1}{\myunpacked}%
	\StrMid{\myunpacked}{1}{#2}% same as \StrLeft{\myunpacked}{#2}
}
\usepackage{amsmath}
\usepackage{amsfonts}
\usepackage{amssymb}
\usepackage{float}
\usepackage{graphicx}
\usepackage{booktabs}
\usepackage{array}
\usepackage[colorlinks=true,citecolor=black,linkcolor=black,filecolor=magenta,urlcolor=cyan]{hyperref}
\begin{document}
	% paste in titling stuff
	\title{Contribution Title\thanks{Supported by organization x.}}
	%
	%\titlerunning{Abbreviated paper title}
	% If the paper title is too long for the running head, you can set
	% an abbreviated paper title here
	%
	\author{First Author\inst{1}\orcidID{0000-1111-2222-3333} \and
		Second Author\inst{2,3}\orcidID{1111-2222-3333-4444} \and
		Third Author\inst{3}\orcidID{2222--3333-4444-5555}}
	%
	\authorrunning{F. Author et al.}
	% First names are abbreviated in the running head.
	% If there are more than two authors, 'et al.' is used.
	%
	\institute{Princeton University, Princeton NJ 08544, USA \and
		Springer Heidelberg, Tiergartenstr. 17, 69121 Heidelberg, Germany
		\email{lncs@springer.com}\\
		\url{http://www.springer.com/gp/computer-science/lncs} \and
		ABC Institute, Rupert-Karls-University Heidelberg, Heidelberg, Germany\\
		\email{\{abc,lncs\}@uni-heidelberg.de}}
	%
	\maketitle              % typeset the header of the contribution
	%
	
	\begin{abstract}\loremnchars[3]{200}\keywords{First keyword  \and Second keyword \and Another keyword.} \end{abstract}
%	\tableofcontents
	\section{Text Modifiers}
	Try typing:
	\begin{itemize}
	\item\begin{verbatim*}\emph{}\end{verbatim*} for italics
	\item\begin{verbatim*}\textbf{}\end{verbatim*} for bold
	\item\begin{verbatim*}\textsc{}\end{verbatim*} for small caps
	\end{itemize}
	\blindtext
\section{\LaTeX   Math}
Math can be delimited by \$ signs, or in special math environments! Lets try doing \emph{inline math}, \emph{display style math}, and a \emph{multiline} math environment. 
\subsection{Doing Math}
\loremnchars[5]{14}  
% $\sum{\alpha^{\frac{\pi}{\sin{e}}}}$ 
\loremnchars[5]{255}

%$\dot { \omega } _ { n } ^ { \prime \prime }$, the source term is defined as:
% or do naver stokes
% try $$ and begin equation
% $$\dot { \omega } _ { n } ^ { \prime \prime } = \frac { 1 } { N _ { A } } \left[ \frac { 2 } { C _ { \min } } N _ { A } k _ { 1 } ( T ) \left[ C _ { 2 } H _ { 2 } \right] - 2 C _ { a } \left( \frac { 6 M _ { s } } { \pi \rho _ { s } } \right) ^ { 1 / 6 } \left( \frac { 6 \kappa T } { \rho _ { s } } \right) ^ { 1 / 2 } C _ { s } ^ { 1 / 6 } [ \rho n ] ^ { 11 / 6 } \right] \left[ \frac { k m o l } { m ^ { 3 } * s } \right]$$

Our Hypothesis can be defined as:
% Use Align environment!

\section{Tables}
\begin{table}[H]
	\caption{A table}
	\label{tb:something}
	\begin{center}
\begin{tabular}{llr}  
	\toprule
	\multicolumn{2}{c}{Item} \\
	\cmidrule(r){1-2}
	Animal    & Description & Price (\$) \\
	\midrule
	Gnat      & per gram    & 13.65      \\
	&    each     & 0.01       \\
	Gnu       & stuffed     & 92.50      \\
	Emu       & stuffed     & 33.33      \\
	Armadillo & frozen      & 8.99       \\
	\bottomrule
\end{tabular}
\end{center}
\end{table}
\section{Figures}
\begin{figure}[H]
	\label{fig:another}
	\begin{center}
		\includegraphics[scale=.2]{Logo}
	\end{center}
\caption{The SMU Logo}
	\end{figure}
\section{Cross Referencing and Table of Contents!}
 In Table \ref{tb:something} we see something. \\ In Figure \ref{fig:another}, we see another thing
 \section{Citations}
 This was a work of spending hours combing through the knitr \cite{knitr} documentation
  \bibliographystyle{splncs04}
  \bibliography{thebib}
 

  
\end{document}